% !TEX program = xelatex
\documentclass{ctexart}
\usepackage{xeCJK}
\usepackage{multicol}   % 分栏
\usepackage{mwe}        % 随机文本
\usepackage[a4paper, inner=3cm, outer=3cm, top=3cm, bottom=3cm, bindingoffset=0cm]{geometry}
\usepackage{fancyhdr}
\usepackage{zhnumber}   % 中文日期

\renewcommand\refname{参考文献}
\renewcommand{\abstractname}{\textbf{\large {摘\quad 要}}} %更改摘要二字的样式

% 页眉控制
\fancyhead{}
\fancyhead[HL]{心理学漫谈结课论文}
\fancyhead[HR]{\zhdate*{2019/1/13}}
\fancyfoot[FC]{\thepage}
\setlength{\headsep}{5pt}

\title{大学生“朋友圈嫉妒”的心理分析}
\date{}
\author{(电子科技大学 20170003 班,成都,611731)}

\begin{document}

\pagestyle{fancy}

\maketitle
\thispagestyle{fancy}

{\bfseries 摘要:}随着社交网络渗透到人们生活的方方面面,刷朋友圈似乎已称为大学生生活不可缺少的一部分。在朋友圈中,大学生可以看到和分享自己与好友生活、知识的点点滴滴,如旅游沿途的美景、大快朵颐的没事,精心拍摄的美照等……可是当人们浏览朋友圈时有时会羡慕他人拥有而自己所没有的物质、经历、和受关注程度等,甚至会进而产生嫉妒心理,给自己的生活带来消极的体验、影响自己的社交关系等。本篇称之为“朋友圈嫉妒”,并就其产生的心理原因、影响和解决方法进行了深入分析。

{\bfseries 关键字:}朋友圈;嫉妒;心理分析

\begin{multicols}{2} % 双栏控制
\section{第一段}
这里是正文。\lipsum[1]

\section{第二段}
这里是正文 \lipsum[2]
\end{multicols}

\begin{thebibliography}{}
	
	\bibitem{Li10}
	Chua, T. H. H., \& Chang, L. (2016). Follow me and like my beautiful selfies: Singapore teenage girls’ engagement in self-presentation and peer comparison on social media. Computers in Human Behavior, 55(Part A), 190–197. https://doi.org/10.1016/j.chb. 2015.09.011. 
	
	\bibitem{Zhang10}
	R. Zhang, and L. Liu, Security Models and Requirements for Healthcare Application Clouds, in: Processing of  Cloud 2010, pp. 268-275, 2010.
\end{thebibliography}

\end{document}